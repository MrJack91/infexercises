\documentclass[10pt]{article}
\usepackage[german]{babel}
\usepackage[utf8]{inputenc}
\usepackage{amssymb}
\usepackage{listings}
\usepackage{enumitem}
\usepackage{fancyhdr}
\usepackage{titling}
\usepackage{pgf}
\usepackage{tikz}
\usepackage{array}
\usepackage{ragged2e}

\usepackage{geometry}
%\usepackage{siunitx}
%\usepackage{amsmath}


\usetikzlibrary{arrows,automata}
% \usepackage[latin1]{inputenc}

\title{Informatik 3 Übung - Teil 2\vspace{-2ex}}
\author{Daniel Brun, Michael Hadorn\vspace{-2ex}}

\setlength{\droptitle}{-6em}     % Eliminate the default vertical space
% \addtolength{\droptitle}{-4pt}   % Only a guess. Use this for adjustment

% \newcolumntype{P}[1]{>{\centering\hspace{0pt}}p{#1}}

\pagestyle{fancy}
% clear any old style settings
\fancyhead{}
\fancyfoot{}

\lhead{ZHAW: Informatik 3}
\rhead{Daniel Brun, Michael Hadorn, Inf 3b}
\fancyfoot[LE,RO]{\thepage}

\usepackage{color}


% \geometry{top=3cm,left=2cm,right=2cm,bottom=3cm}
\geometry{top=3cm,left=2cm,right=2cm,bottom=3cm}


% This sets page margins to .5 inch if using letter paper, and to 1cm
% if using A4 paper. (This probably isn't strictly necessary.)
% If using another size paper, use default 1cm margins.

%\ifthenelse{\lengthtest { \paperwidth = 11in}}
%	{ \geometry{top=.5in,left=.5in,right=.5in,bottom=.5in} }
%	{\ifthenelse{ \lengthtest{ \paperwidth = 297mm}}
%		{\geometry{top=1cm,left=1cm,right=1cm,bottom=1cm} }
%		{\geometry{top=1cm,left=1cm,right=1cm,bottom=1cm} }
%	}

%\setlength{\parindent}{0pt}
%\setlength{\parskip}{0pt plus 0.5ex}


\begin{document}
\maketitle

% Aufgabe 1  	a: OK	b: OK	c: OK	d: 
\section*{Aufgabe 1}
Gegeben sei ein Prozessor mit einer Taktzykluszeit von 1.25 GHz und einem CPI-Wert von 1.45 (der Prozessor verfügt über keine Pipeline). Ein Programm benötigt zur Ausführung 150‘000 Befehle.

\begin{enumerate}[label=\alph*)]
	\item 
	\textit{Wie lang ist die ungefähre Ausführungszeit des	Programms? Lösung: (3 Punkte)}\\
	$ (1 / 1.25 GHz)*(150'000 * 1.45) = 1.74 * 10^{-4} s$
	
	\item
	\textit{Wieso ist der berechnete Wert nur ein Näherungswert? Lösung: (2 Punkte)}\\
	Da der CPI-Wert der Durchschnitt der Zyklen eines Befehles definiert, kann man daraus nicht folgern, dass sich der Zeit-Aufwand der Ausführung genau aus diesem ableiten lässt. Wenn es z.B. mehrere Befehle gibt, die mehrere Zyklen benötigen, so wird die Ausführung länger dauern.
	\item
	\textit{Der Prozessor wird durch einen leistungsfähigeren Prozessor mit 0.4 ns Taktzykluszeit und einem CPI-Wert von 1.8 ersetzt. Wie lang ist nun die Ausführungszeit des Programms? Lösung: (2 Punkte)}\\
	% $ 0.4 ns *(150'000 * 1.8) = 10'800 ns = 1.08 * 10^{-5} s$
	
	%TODO
	% \textbf{Nicht?}\\	Hmm...Dann wäre es aber 1.08 * 10^{-8}, da ansonsten der Wert zu gross ist..
	$ (150'000*1.8)*(0.4*10^{-9}) = 0.000108 s$\\
	$1.08*10^{-4} s$ (mit korrekten Signifikantenstellen)
			
	\item
	\textit{Der Prozessor (von c) wird um 10\% übertaktet ("`overclocking"'). Die erzielte Leistungssteigerung beträgt in der Realität aber nur knapp 5\%. Wieso? 	Lösung: (2 Punkte)}\\
	
	%TODO vielleicht ;) Stern hat im Unterricht was erwähnt, in den Folien steht es leider nicht. Der Buss könnte es sein, bin mir aber nicht sicher.
	Weil die Busse nicht mithalten können. Das heisst die CPU muss öfters auf die Daten warten und kann nichts machen. Die Ausführung dauert darum nicht schneller.
	
\end{enumerate}
\newpage

% Aufgabe 2		a:	OK b: OK
\section*{Aufgabe 2}
Gegeben sei ein einfacher Prozessor ohne Pipelining mit einer Wortbreite von 2 Byte (für Daten und Befehle). 
\begin{enumerate}[label=\alph*)]
	\item 
	\textit{Welchen Wert beinhaltet der Befehlszähler jeweils nach Ausführung der jeweiligen Befehle der folgenden Befehlssequenz (der Initialwert sei 24'048 für den ersten Befehl): Ladebefehl, Ladebefehl, Addition, unbedingter Sprung um -12, Speicherbefehl, unbedingter Sprung um + 8, Addition... ? Lösung: (4 Punkte)} \\
	%TODO stimmt das so? Wieso erhöht jeder Ladebefehl den Befehlszähler?
	% Da ja alles ein eigenständiger Befehl ist, habe ich angenommen, dass dann auch der Zähler jedesmal erhöht wird. Evtl. können wir ihn im Unterricht nochmals fragen.
	Initial: 24'048\\
	1. Ladebefehl: 24'049\\
	2. Ladebefehl: 24'050\\
	3. Addition: 24'051\\
	4. Unbedingter Sprung: 24'039 \\
	5. Speicherbefehl: 24'040\\
	6. Unbedingter Sprung: 24'048 \\
	...
	\item
	\textit{Was sehen Sie als Informatiker sofort? Lösung: (2 Punkte)} \\
	%TODO: Evtl. ist es auch ein Unterprogramm oder eine normale Schleife...
	Es Handelt sich um eine Endlosschleife / Nicht-Terminierenden-Code.
		
\end{enumerate}


% Aufgabe 3		a:	b:
\section*{Aufgabe 3}
Gegeben sei ein Prozessor mit 4-stufiger Pipeline (die vier Stufen, wie in der Vorlesung angegeben) und folgender Ausschnitt einer Programmabfolge: 
..., Load, Sprung, Addition, ODER-Operation, Store, Subtraktion, Sprung, AND-Operation, ...
\begin{enumerate}[label=\alph*)]
	\item 
	\textit{Skizzieren Sie graphisch eine (mögliche) Ausführungsabfolge, unter der Annahme, das beim 1. Sprung zu einer nicht vorhergesehenen Adresse gesprungen wird ("`branch prediction"' war falsch).} \\
	\item 
	\textit{Beschreiben Sie in Ihren Worten, was ein "`pipeline flush"' bedeutet. Lösung: (2 Punkte) } \\
	
\end{enumerate}
\newpage

% Aufgabe 4		a: OK
\section*{Aufgabe 4}
Eine effektive Möglichkeit der Leistungssteigerung bei Prozessoren ist Pipelining.
\begin{enumerate}[label=\alph*)]
	\item 
	\textit{Begründen 	Sie, warum eine n-stufige Pipeline nicht automatisch zu einer n-fachen Leistungssteigerung führt, selbst wenn es gelingt, die Zykluszeit auf 1/n zu reduzieren ("`perfekte Gleichverteilung"' der Stufen - in der Praxis eigentlich nicht realisierbar). Lösung:(4 Punkte)} \\	
	Beim Pipelining wird nach dem Laden des Befehls bereits der nächste Geladen, dies setzt voraus, dass der nachfolgende Befehl mit einer hohen Genauigkeit vorausgesagt werden kann ("`Branch Prediction"'). Durch die spezielle Architektur, die für das Pipelining notwendig ist, erhöht sich die Komplexität innerhalb des Prozessores um ein vielfaches. Durch die zusätzlich notwendigen Mechanismen zur Kontrolle, Abschirmung, etc. wird auch Leistung benötigt, was der Grund dafür ist, dass keine n-fache Leistungssteigerung erzielt werden kann.
\end{enumerate}
\newpage

% Aufgabe 5		a:	b:	c:	d:
\section*{Aufgabe 5}
Gegeben sei ein Prozessor ohne Pipeline mit der "`bekannten"' Befehlsabarbeitung (siehe Vorlesung) und einer Zykluszeit von 20MHz. Ein Analyse hat ergeben, dass die einzelnen Teilschritte sehr unterschiedliche Zeit erfordern:\\
z. B. "`Befehl laden"' $\leq$ 10ns, "`Register lesen"' $\leq$ 3ns, "`Rechenoperation durchführen"' $\leq$ 5ns, "`Speicherzugriff"' $\leq$ 20ns und "`Register schreiben"' $\leq$ 5ns, ... \\
Sie implementieren denselben Prozessor mit einer 5-stufigen Pipeline (die bisherigen Teilschritte erfordern gleich viel Zeit). 
\begin{enumerate}[label=\alph*)]
	\item 
	\textit{Wie gross ist die Zykluszeit des neuen Prozessors? Lösung: (3 Punkte) } \\	
	\item 
	\textit{Um wie viel schneller wird nun ein Befehl maximal ausgeführt? Lösung: (3 Punkte)} \\	
	\item 
	\textit{Um wie viel schneller wird ein Programm maximal ausgeführt? Lösung: (2 Punkte) } \\
	\item 
	\textit{Wie	könnte eine "`bessere"' Pipeline-Struktur entwickelt werden? Lösung: (2 Punkte)  } \\
\end{enumerate}

% Aufgabe 6		a:
\section*{Aufgabe 6}
Eine effektive Möglichkeit der Leistungssteigerung bei Prozessoren ist Pipelining.
\begin{enumerate}[label=\alph*)]
	\item 
	\textit{Recherchieren Sie die Pipelinestruktur und Kenndaten von zwei aktuellen Mikroprozessoren. Bitte Kenndaten angeben und Struktur skizzieren, inkl. Quellenangabe. Lösung: (6 Punkte) } \\	
\end{enumerate}
\newpage
\newpage

\end{document}