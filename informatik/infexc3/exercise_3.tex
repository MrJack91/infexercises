\documentclass[10pt]{article}
\usepackage[german]{babel}
\usepackage[utf8]{inputenc}
\usepackage{amssymb}
\usepackage{listings}
\usepackage{enumitem}
\usepackage{fancyhdr}
\usepackage{titling}
\usepackage{pgf}
\usepackage{tikz}
\usepackage{array}
\usepackage{ragged2e}

\usetikzlibrary{arrows,automata}
% \usepackage[latin1]{inputenc}

\title{Informatik 3 Übung - Teil 3\vspace{-2ex}}
\author{Daniel Brun, Michael Hadorn\vspace{-2ex}}

\setlength{\droptitle}{-6em}     % Eliminate the default vertical space
\addtolength{\droptitle}{-4pt}   % Only a guess. Use this for adjustment

\newcolumntype{P}[1]{>{\centering\hspace{0pt}}p{#1}}

\pagestyle{fancy}
% clear any old style settings
\fancyhead{}
\fancyfoot{}

\lhead{ZHAW: Informatik 3}
\rhead{Daniel Brun, Michael Hadorn, Inf 3b}
\fancyfoot[LE,RO]{\thepage}

\usepackage{color}

\begin{document}
\maketitle

% Aufgabe 1  	a: -	b: -
\section*{Aufgabe 1}
Kontrollstrukturen, wie z. B. Schleifen in höheren Programmiersprachen, werden in Maschinensprache über Sprungbefehle realisiert.
\begin{enumerate}[label=\alph*)]
	\item
		\textit{Skizzieren Sie graphisch (schematisch), wie eine For- und eine While-Schleife, die 10-mal \(n \ge 0\) durchlaufen wird, umgesetzt werden könnte. (4 Punkte)}
		
		%TODO mha: wie genau soll das aussehen? Notation? - Ich begreife nicht wie wir den Counter als while abbilden sollen. Habe mir Skizzen gezeichnet, aber verstehe es nicht...
		%while -> p.33
			% 1) Bedingung berechnen & in Register schreiben
			% 2) Test Register auf null
				% 2a) positiv: Kein Sprung, Ausführung, Rücksprung zu Schritt 1
				% 2b) negativ: Sprung zu 1. Befehl nach Schleife
	
	
	%TODO mha: Hier ein Entwurf: https://docs.google.com/drawings/d/1G_d7kC5rVWmTCPEIGl2dgUg-T0XShsYCb4RZekhaa88/edit
		%for (Lösung für zweite Frage)
				% 1) Schleifenzähler initialisieren (> 0) und in Register speichern
				% 2) Befehle ausführen
				% 3) Zähler um 1 reduzieren
				% 4) Test Schleifenzähler auf null
					% 4a) negativ: Sprung zu 1. Befehl 
					% 4b) positiv: Kein Sprung
	
	\item
		\textit{Mit welcher Einschränkung ist es möglich, die FOR-Schleife mit nur einem Sprungbefehl zu realisieren? Skizzieren Sie auch diesen Fall. (2 Punkte)}
		% Wenn zu Beginn der Schleife keine Prüfung erfolgt	
		%TODO mha: also wie im Beispiel des For's oben (in Aufgabe 1b)?
		
		
\end{enumerate}


% Aufgabe 2	 	a: OK	b: OK
\section*{Aufgabe 2}
Das Steuerwerk eines Rechners dekodiert die Befehle aus dem OP- Code bzw. dem Maschinencode; für Benutzer sind Befehle mit mnemonische Symbolen leichter zu lesen (und schreiben). Gehen Sie vom Befehlssatz für den "`Mini-Power-PC"' aus.
\begin{enumerate}[label=\alph*)]
	\item
		\textit{Geben Sie für die folgenden Befehle mit mnemonische Symbolen den Maschinencode an: (4 Punkte)}
		
			\begin{tabular}[h]{l l}
				LWDD 1, \#240 & 0100 0100 1111 0000\\
				ADDD \#62 & 1000 0000 0011 1110\\
				Not & 0000 0000 1000 0000\\
				BCD \#15 & 0011 1000 0000 1111\\
			\end{tabular}
	
	\item
		\textit{"'Dekodieren"' Sie die folgenden Befehle in Maschinencode so weit es möglich ist (mnemonische Symbol und Beschreibung): (4 Punkte)}

			\begin{tabular}[h]{l l}
				0001 1111 1110 1111 & BC R3\\
				0101 1010 0000 0000 & LWDD R2, \#512 \\
				0000 1011 0001 1010 & OR R2 \\
				0110 0001 1111 0110 & SWDD Akku, \#502\\ %TODO mha: R0 mit Akku ersetzt, stimmt so oder? DBRU: R0 == AKKU, kommt von daher nicht drauf an ;-)
			\end{tabular}

\end{enumerate}

% Aufgabe 3		 	a: OK	b: OK
\section*{Aufgabe 3}
Der Befehlssatz für den "`Mini-Power-PC"' ist sehr klein / eingeschränkt. Sie haben gelernt, dass Zugriffe auf den Arbeitsspeicher sehr langsam sind (zum Teil deutlich mehr als 100 Zyklen).

\begin{enumerate}[label=\alph*)]
	\item
		\textit{Welchen Befehlstyp würden Sie auf jeden Fall ergänzen, um die Code-Ausführung erheblich zu beschleunigen? Lösung: (3 Punkte)}
		
		Man sollte direkt vom Akku den Wert in ein anderes Register schreiben können und umgekehrt. Sonst muss man diesen Wert immer via Arbeitsspeicher hin und her kopieren.
		
	\item
		\textit{Kann mit dem Befehlssatz ein Arbeitsspeicher von 16 KiB genutzt werden? Antwort bitte begründen. Lösung: (3 Punkte)}
		
		Nein, für die Adressierung stehen jeweils nur 10 Bit zur Verfügung, damit können ohne zusätzliche Komponenten oder einem anderen Aufbau (andere Unterteilung des Arbeitsspeicher) nicht mehr als 1 KiB adressiert werden.
			
\end{enumerate}

\newpage

% Aufgabe 4		 	a: OK	b: -	c: -
\section*{Aufgabe 4}
Gegeben sei der Befehlssatz für den "`Mini-Power-PC"'. Die Aufgabe Summe=a+4*b+8*c soll über ein Programm für den „Mini-Power-PC berechnet werden.

\begin{enumerate}[label=\alph*)]
	\item
		\textit{Schreiben Sie den Programm-Code mit mnemonische Symbolen (in Assembler) (6 Punkte)}
		
		\begin{tabular}[h]{l | l}
			\textbf{Befehl} & \textbf{Comment}\\
			\hline
			\#100 = a & Init operation var a\\
			\#102 = b & Init operation var b\\
			\#104 = c & Init operation var c\\
			\#106 = s & Init temp/result var \\
			& \\
			LWDD R00, \#104 & akku: load var c (to multiplicate with 8)\\
			SLL & value = 2x c \\ %TODO: SLL oder LS (SLA)?
			SLL & value = 4x c\\
			SLL & value = 8x c\\
			SWDD R00, \#106 & write akku to temp var\\
			& \\
			LWDD R00, \#102 & akku: read var b \\
			SLL & value = 2x b\\
			SLL & value = 4x b\\
			& \\
			LWDD R01, \#100 & r01: load var a\\ %TODO mha: #102 to 100 gewechselt
			ADD R01 & add akku (4xb) and r01 (a)\\
			& \\
			LWDD R01, \#106	 & r01: load temp (8xc) \\
			ADD R01 & add temp to akku \\
			& \\
			SWDD R00, \#106 & write akku to var c -> result\\
			& \\
			END & \\
			
		\end{tabular}

	\item
			\textit{Übersetzen Sie den Code in Maschinencode (4 Punkte)}
			
			%TODO mha: Erst wenn der erste Teil stimmt.
			
	\item
			\textit{Berechnen Sie mit Hilfe Ihres Programms: (4 Punkte)}
			
			- Summe für a=14, b=7 und c=66
			- Summe für a=25, b=-14 und c=-123
			- Summe für a=-125, b=10’000 und c=16
			- Summe für a=1000, b=10’000 und c=-2’000
\end{enumerate}

\end{document}