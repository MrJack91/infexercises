\documentclass[10pt]{article}
\usepackage[german]{babel}
\usepackage[utf8]{inputenc}
\usepackage{amssymb}
\usepackage{listings}
\usepackage{enumitem}
\usepackage{fancyhdr}
\usepackage{titling}
\usepackage{pgf}
\usepackage{tikz}
\usepackage{array}
\usepackage{ragged2e}

\usetikzlibrary{arrows,automata}
% \usepackage[latin1]{inputenc}

\title{Informatik 3 Übung - Teil 3\vspace{-2ex}}
\author{Daniel Brun, Michael Hadorn\vspace{-2ex}}

\setlength{\droptitle}{-6em}     % Eliminate the default vertical space
\addtolength{\droptitle}{-4pt}   % Only a guess. Use this for adjustment

\newcolumntype{P}[1]{>{\centering\hspace{0pt}}p{#1}}

\pagestyle{fancy}
% clear any old style settings
\fancyhead{}
\fancyfoot{}

\lhead{ZHAW: Informatik 3}
\rhead{Daniel Brun, Michael Hadorn, Inf 3b}
\fancyfoot[LE,RO]{\thepage}

\usepackage{color}

\begin{document}
\maketitle

% Aufgabe 1  	a: -	b: -
\section*{Aufgabe 1}
Kontrollstrukturen, wie z. B. Schleifen in höheren Programmiersprachen, werden in Maschinensprache über Sprungbefehle realisiert.
\begin{enumerate}[label=\alph*)]
	\item
		\textit{Skizzieren Sie graphisch (schematisch), wie eine For- und eine While-Schleife, die 10-mal \(n \ge 0\) durchlaufen wird, umgesetzt werden könnte. (4 Punkte)}
	%while
	% 1) Bedingung berechnen & in Register schreiben
	% 2) Test Register auf null
	% 2a) positiv: Kein Sprung, Ausführung, Rücksprung zu Schritt 1
	% 2b) negativ: Sprung zu 1. Befehl nach Schleife
	
	%for (Lösung für zweite Frage)
	% 1) Schleifenzähler initialisieren (> 0) und in Register speichern
	% 2) Befehle ausführen
	% 3) Zähler um 1 reduzieren
	% 4) Test Schleifenzähler auf null
	% 4a) negativ: Sprung zu 1. Befehl 
	% 4b) positiv: Kein Sprung
	
	\item
		\textit{Mit welcher Einschränkung ist es möglich, die FOR-Schleife mit nur einem Sprungbefehl zu realisieren? Skizzieren Sie auch diesen Fall. (2 Punkte)}
		% Wenn zu Beginn der Schleife keine Prüfung erfolgt	
			
\end{enumerate}


% Aufgabe 2	 	a: - OK	b: - OK
\section*{Aufgabe 2}
Das Steuerwerk eines Rechners dekodiert die Befehle aus dem OP- Code bzw. dem Maschinencode; für Benutzer sind Befehle mit mnemonische Symbolen leichter zu lesen (und schreiben). Gehen Sie vom Befehlssatz für den "`Mini-Power-PC"' aus.
\begin{enumerate}[label=\alph*)]
	\item
		\textit{Geben Sie für die folgenden Befehle mit mnemonische Symbolen den Maschinencode an: (4 Punkte)}
		- LWDD 1, #240 	: 0100010011110000
		- ADDD #62		: 1000000000111110
		- Not			: 0000000010000000
		- BCD #15		: 0011100000001111
	
	\item
		\textit{"'Dekodieren"' Sie die folgenden Befehle in Maschinencode so weit es möglich ist (mnemonische Symbol und Beschreibung): (4 Punkte)}

		- 00011111 11101111 : BC R3
		- 01011010 00000000 : LWDD R2,#512 
		- 00001011 00011010 : OR R2
		- 01100001 11110110 : SWDD R0, #502

\end{enumerate}

% Aufgabe 3		 	a: -	b: - OK
\section*{Aufgabe 3}
Der Befehlssatz für den "`Mini-Power-PC"' ist sehr klein / eingeschränkt. Sie haben gelernt, dass Zugriffe auf den Arbeitsspeicher sehr langsam sind (zum Teil deutlich mehr als 100 Zyklen).

\begin{enumerate}[label=\alph*)]
	\item
		\textit{Welchen Befehlstyp würden Sie auf jeden Fall ergänzen, um die Code-Ausführung erheblich zu beschleunigen? Lösung: (3 Punkte)}
		%Wert aus dem Akku in ein Register schreiben -> sonst muss man immer über den Arbeitsspeicher gehen
		
	\item
		\textit{Kann mit dem Befehlssatz ein Arbeitsspeicher von 16 KiB genutzt werden? Antwort bitte begründen. Lösung: (3 Punkte)}
		Nein, für die Adressierung stehen jeweils nur 10 Bit zur Verfügung, damit können ohne zusätzliche Komponenten oder eine anderen Aufbau (Unterteilung Arbeitsspeicher) nicht mehr als 1 KiB adressiert werden.
			
\end{enumerate}

% Aufgabe 4		 	a: -	b: -	c: -
\section*{Aufgabe 4}
Gegeben sei der Befehlssatz für den "`Mini-Power-PC"'. Die Aufgabe Summe=a+4*b+8*c soll über ein Programm für den „Mini-Power-PC berechnet werden.

\begin{enumerate}[label=\alph*)]
	\item
		\textit{Schreiben Sie den Programm-Code mit mnemonische Symbolen (in Assembler) (6 Punkte)}
		#100 = a
		#102 = b
		#104 = c
		#106 = s
		
		LWDD R00, #104
		SLL %TODO: SLL oder LS?
		SLL
		SLL
		SLL
		SWDD R00, #106
		
		LWDD R00, #102
		SLL
		SLL
		
		LWDD R01, #102	
		ADD R01
		
		LWDD R01, #106	
		ADD R01
				
		SWDD R00, #106
	
		END
		
	\item
			\textit{Übersetzen Sie den Code in Maschinencode (4 Punkte)}
			
	\item
			\textit{Berechnen Sie mit Hilfe Ihres Programms: (4 Punkte)}
			
			- Summe für a=14, b=7 und c=66
			- Summe für a=25, b=-14 und c=-123
			- Summe für a=-125, b=10’000 und c=16
			- Summe für a=1000, b=10’000 und c=-2’000
\end{enumerate}

\end{document}