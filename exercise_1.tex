\documentclass[10pt]{article}
\usepackage[german]{babel}
\usepackage[utf8]{inputenc}
\usepackage{amssymb}
\usepackage{listings}
\usepackage{enumitem}
\usepackage{fancyhdr}
\usepackage{titling}
\usepackage{pgf}
\usepackage{tikz}
\usetikzlibrary{arrows,automata}
% \usepackage[latin1]{inputenc}

\title{Informatik 3 Uebung - Teil 1\vspace{-2ex}}
\author{Daniel Brun, Michael Hadorn\vspace{-2ex}}

\setlength{\droptitle}{-6em}     % Eliminate the default vertical space
\addtolength{\droptitle}{-4pt}   % Only a guess. Use this for adjustment


\pagestyle{fancy}
% clear any old style settings
\fancyhead{}
\fancyfoot{}

\lhead{ZHAW: Informatik 3}
\rhead{Daniel Brun, Michael Hadorn, Inf 3b}
\fancyfoot[LE,RO]{\thepage}

\usepackage{color}

\begin{document}
\maketitle

% TODO: Umlaute korrekt anzeigen (auch im Titel)

% Aufgabe 1
\section{Aufgabe}
Beispiele fu?r klassische Prozessoren sind: Intel 4004, Intel 8008, Intel 8088, Intel 8086, Intel 80286, Intel 80386, Motorola 68000, Z80, MOS 6502, PowerPC 970, PDP-11, CDP1802.

\begin{enumerate}[label=\alph*)]
	\item 
	\textit{Geben Sie das Erscheinungsjahr sowie die intern verwendeten Wortbreite an (4 Punkte).}
	
	\item
	\textit{Wie viele verschiedene Befehle ko?nnen damit dargestellt werden? (2 Punkte)}
\end{enumerate}


% Aufgabe 2
\section{Aufgabe}
Der ?Pufferu?berlauf? geho?rt zu den ha?ufigsten Sicherheitslu?cken in Programmen (mit Computern mit der Von-Neumann-Architektur).
\begin{enumerate}[label=\alph*)]
	\item 
	\textit{Beschreiben Sie kurz informell, warum die klassische Harvard- Architektur besser gegen diesen schu?tzt (gegenu?ber der Von- Neumann-Architektur). (4 Punkte)}
	
	\item
	\textit{Ist Ihre Argumentation auch bei der Super-Harvard-Architektur allgemein korrekt? (2 Punkte)}
\end{enumerate}


% Aufgabe 3
\section{Aufgabe}
Wortbreiten
\begin{enumerate}[label=\alph*)]
	\item 
	\textit{Kann ein Prozessor mit geringer Wortbreite auch Werte (bzw. Worte) berechnen, die breiter sind? Zum Beispiel ein Prozessor mit 8-Bit- Wortbreite auch 16- oder 32-Bit-Wo?rter. Falls ja, wie ko?nnte ein solches Verfahren aussehen? (4 Punkte)}
	
\end{enumerate}
\newpage





% Aufgabe 4
\section{Aufgabe}
Architektur
\begin{enumerate}[label=\alph*)]
	\item 
	\textit{Wieso ko?nnen der Motorola 68000 und die Intel-Prozessoren 8088, 8086 und 80286 mehr als 65 KB Hauptspeicher adressieren?
(4 Punkte)	}
	

	
	\item
	\textit{Was unterscheidet den Motorola 68000 von der Architektur des Intel x86? Welche Vor- und Nachteile ergeben sich daraus? (4 Punkte)}
\end{enumerate}
\newpage



\end{document}