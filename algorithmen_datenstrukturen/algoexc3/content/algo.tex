%  ersetzen: (stehen lassen für jeweiligen replace nach copy&paste vom pdf)
%	* ä mit ä
%	* ö mit ö
%	* ü mit ü
%	* anführungszeichen...

\chapter*{Algorithmen und Datenstrukturen - Aufgabenserie 3}
\begin{flushright}
by Daniel Brun und Michael Hadorn, Inf I3b
\end{flushright}
\section*{Aufgabe 1 – nochmal Komplexität}

Vergleichen Sie das Wachstum untenstehender Programme und geben Sie diese in der Landau-Notation an.\\

\noindent
while x < n do\\
	x := x + 1\\
	n := n - 1\\	
end\\

\noindent
while x < n do\\
	n := n / 2\\
	x := 2 * x\\
end\\

\subsection*{Algorithmus 1}
Der erste Algorithmus hat eine lineare Laufzeit ($O(n)$). Die Laufzeit bei einem höheren Input bleibt im Verhältnis immer gleich.

\subsection*{Algorithmus 2}
Der zweite Algorithmus hat eine logarithmische Laufzeit ($O(log(n))$). Bei einer Verdoppelung des Inputs nimmt die Laufzeit im Verhältnis nur minim zu.

\newpage

\section*{Aufgabe 2 – Listen}
Implementieren Sie den Datentyp einer einfach verketteten Liste (mit integer Datenfeldern). Die Listen sollen folgende Funktionalitäten aufweisen:\\

\begin{itemize}
	\item
		Das erste Element der Liste auslesen.
	\item
		Das letzte Element der Liste auslesen.
	\item
		Ein Objekt am Anfang der Liste hinzufügen.
	\item
		Ein Objekt am Schluss der Liste hinzufügen.
	\item
		Anzahl Elemente der Liste zurückgeben.
	\item
		Mit einer anderen Liste vergleichen.
	\item
		Abfragen ob ein bestimmtes Objekt in der Liste vorkommt.
\end{itemize}

\lstinputlisting[language=Java]{content/DevAlgoexc3/src/ch/zhaw/mhdb/ad/List.java}

\lstinputlisting[language=Java]{content/DevAlgoexc3/src/ch/zhaw/mhdb/ad/ListEntry.java}
\newpage

\section*{Aufgabe 3 – Menge}
Benutzen Sie Ihre Implementation von Listen aus der ersten Aufgabe und implementieren Sie den Datentyp einer Menge mit folgenden Funktionalitäten:

\begin{itemize}
	\item
		Abfrage ob ein bestimmtes Element zur Menge gehört.
	\item
		Die Menge als String von der Form {x1, x2, . . . } zurückgeben.
	\item
		Ein Element hinzufügen.
	\item
		Mit einer anderen Menge vereinigen.
	\item
		mit einer anderen Menge schneiden.
	\item
		Anzahl Elemente der Menge abfragen. Beachten Sie, dass mehrfach vorkommen- de Elemente nur einmal gezählt werden sollen.
	\item
		Mit einer anderen Menge vergleichen. Beachten Sie, dass beim Vergleich von Mengen die Reihenfolge und Wiederholungen keine Rolle spielen.
		
\end{itemize}

\lstinputlisting[language=Java]{content/DevAlgoexc3/src/ch/zhaw/mhdb/ad/Menge.java}