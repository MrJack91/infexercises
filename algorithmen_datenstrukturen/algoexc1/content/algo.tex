\chapter*{Algorithmen und Datenstrukturen - Aufgabenserie 2}
\begin{flushright}
by Daniel Braun und Michael Hadorn, Inf I3b
\end{flushright}
\section*{Aufgabe 1 - Grundlagen zur Laufzeitkomplexität}
\begin{enumerate}
	\item % Aufgabe 1.1
	Ein Algorithmus kann die gleiche Aufgabe auf verschiedene Wege lösen. Dabei gibt es effizientere und ineffizientere Möglichkeiten.
	Beispiele die die Laufzeit beeinflussen können sind:
	\begin{enumerate}
		\item \textbf{Länge des Inputs Textes}\\
			Es kann sein, dass der eine Algorithmus für kürzere Inputs schneller ist, jedoch sich für längere Inputs nicht eignet und umgkehrt.
		
		\item \textbf{Komplexität}\\
			Ein Algorithmus kann ein komplexes Problem effizienter oder eben nicht so effizient lösen. Das Resultat ist das selbe, aber der ineffiziente dauert länger.
		
		% \item \textbf{}\\
				
		% \item \textbf{}\\
			
		%Anzahl Elemente einer Matrix, Grad eines Polynmos, Läng eines Textes, Rekursion / Verschachtelung,
	\end{enumerate}
	
	\item % Aufgabe 1.2
	Mit der O-Notation. t = O(n).\\
	Dabei können Konstanten vernachlässigt werden, da die sowieso nur einen kleinen Teil der Ausführungszeit benötigen. Der interessante Wachstum ist abhängig vom Input.
	
	\item % Aufgabe 1.3
	Folgende externe Einflüsse können sich auf die Laufzeit des Algorithmuses auswirken:\\
	\begin{enumerate}
		\item \textbf{Hardware}\\
		Prozessor, Arbeits-/ Zwischenspeicher, Systemzeitgeber
		\item \textbf{Betriebssytem}\\
		andere Anwndungs-/ Systemprozesse, Sicherheitseinstellungen, virtueller Speicher (Swap)
		\item \textbf{Programmiersprache}\\
		Interpretation, Bytecode, Maschinencode, ...
		\item \textbf{Compiler}\\
		Optimierungen, Laufzeitoptimierungen, Garage Collection
	\end{enumerate}
	
	\item % Aufgabe 1.4
	90\% der Laufzeit eines Codes wird nur in 10\% des Codes verbracht. Daher sollten diese 10\% optimiert werden und nicht die anderen 90\% des Codes, welche nur selten ausgeführt werden. Um eine Laufzeitanalyse für eine Optimierung vorzunehmen kann man einen Profiler verwenden.
	
	\item % Aufgabe 1.5
	Nein, es kann durchaus sein, dass der Lineare-Algorithmus vor allem zu Beginn mehr Zeit benötigt als ein Exponentieller-Algorithmus. Voraussetzung dafür ist, dass der Exponentielle-Algorithmus klein Anfängt, bzw. einige Zeit benötigt um zu "wachsen".
\end{enumerate}

\section*{Aufgabe 2  Landau Notation}
\begin{enumerate}
	\item
	\textit{Falsch, da $F \in O(G) \land G \in O(2^{2^x}) \implies F \in O(2^{2^x})$}
	\item
	\textit{Wahr, da durch die Definition von O folgendes gilt: $0 \leq f(n) \leq c*g(n), n\geq n_0$ ($c, n_0$) Konstanten. Da $F \in O(2^{2^x})$ ist auch $F \in O(2^x)$ (mit c = 2)}
	\item
	\textit{Unklar, je nach dem wie G aussieht gibt es keine Konstante $c$, sodass $g(n) \leq c * O(1)$}
	\item
	\textit{Wahr, da $F \in O(F) \land F \in O(G) \implies O(F) \subseteq O(G) $}
\end{enumerate}

\newpage

\section*{Aufgabe 3 - Diamant}

%mögliche Optimierung: Erster Loop bei fillWithStars() überspringen, dafür die zwei Werte (jeweils nur die Mitte) vorinitialisieren.
\lstinputlisting[language=Java]{content/src/StarSquare.java}

\begin{landscape}
Feinanalyse des Algorithmus:

\begin{tabular}{l | c | c | c |}
Anweisung & Anzahl & Einzelgewicht & Gewicht \\
\hline
1	int dimension = anArray.length;	&	1	&	& 	1.0		\\
2	int middleLine = ((dimension - 1) / 2);	&	1	&	1.0 + 1.4 + 8.0	& 	10.4	\\
3.	int offset = 0;	&	1	&	& 	1.0	\\
4.	for (int col = 0; col <= middleLine; col++) \{	&	(n -1)/2+2	& 	&  1.5		\\
5.	anArray[middleLine][col] = '*';	&	(n -1)/2+1	&	& 	4.2	\\
6.	anArray[middleLine + offset][col] = '*';	&	(n -1)/2+1	& 1.0 + 4.2	& 	5.2	\\
7.	anArray[middleLine - offset][col] = '*';	&	(n -1)/2+1	& 1.0 + 4.2	& 	5.2	\\
8.	anArray[middleLine][dimension - col - 1] = '*';	&	(n -1)/2+1	& 1.0 + 1.0 + 4.2	& 	6.2	\\
9.	anArray[middleLine + offset][dimension - col - 1] = '*';	&	(n -1)/2+1	&	1.0 + 1.0 + 1.0 + 4.2	& 	7.2	\\
10.	anArray[middleLine - offset][dimension - col - 1] = '*';	&	(n -1)/2+1	&	1.0 + 1.0 + 1.0 + 4.2	& 	7.2	\\
11.	offset++;	&		(n -1)/2+1		&	1.0 + 1.4	& 	2.4	\\
12.	\}
\end{tabular}\\
Laufzeit $t = 1 + 10.4 + 1 + ((n-1)/2 + 2)*1.5 + ((n-1)/2 + 1)*4.2*1 + ((n-1)/2 + 1)*2.4 + 2 * ((n-1)/2 + 1)*5.4 + 1 * ((n-1)/2 + 1)*6.2 + + 2 * ((n-1)/2 + 1)*7.2$\\
$t = 38*(\frac{n-1}{2} + 1) + 1.5*(\frac{n-1}{2} + 2) + 12.4 $\\
\boldmath{$t = 19.75x + 33.65 $}\\
\end{landscape}
\clearpage

\section*{Aufgabe 4 - Fibonacci reloaded}
Gerade Fibanocci Zahlen bis 4'000'000.
Anschliessend die Summe davon.

\begin{center}
% Style changes
\small\renewcommand{\arraystretch}{1.4}
% tabular
\small
\rowcolors{1}{tablerowcolor}{tablebodycolor} % mark even rows grey
\begin{tabularx}{0.25\textwidth}{rr}
\rowcolor{tableheadcolor}
\hline
\textbf{\#Fib} & \textbf{Wert} \\
\hline
3 & 2 \\
6 & 8 \\
9 & 34 \\
12 & \num{144} \\
15 & \num{610} \\
18 & \num{2584} \\
21 & \num{10946} \\
24 & \num{46368} \\
27 & \num{196418} \\
30 & \num{832040} \\
33 & \num{3524578} \\
\hline
\rowcolor{tablesubheadcolor}
\textbf{Summe} & \textbf{\num{4613732}}\\
\hline
\end{tabularx}
\end{center}
