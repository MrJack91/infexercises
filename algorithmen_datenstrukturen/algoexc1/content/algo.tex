\chapter*{Algorithmus und Datenstruktur - Übungsserie}

\section*{Aufgabe 1}


\section*{Aufgabe 4}

\begin{table}[ht]
\centering
\small\renewcommand{\arraystretch}{1.4}  
\rowcolors{1}{tablerowcolor}{tablebodycolor}
%
\captionabove[]{Gerade Fibanocci Zahlen bis 4'000'000.}
\label{tab:IsingModel}
%
\begin{tabularx}{0.5\textwidth}{lXXX}
\hline
\rowcolor{tableheadcolor}
lattice & $d$ & $q$ & $T_\text{mf}/T_c$ \\
\hline
square  & 2 & 4 & 1.763 \\
%
triangular & 2 & 6 & 1.648 \\
%
diamond & 3 & 4 & 1.479 \\
%
simple cubic & 3 & 6 & 1.330 \\
%
bcc & 3 & 8 & 1.260 \\
%
fcc & 3 & 12 & 1.225 \\
\hline
\end{tabularx}
\end{table}

fib to add: 2
fib to add: 8
fib to add: 34
fib to add: 144
fib to add: 610
fib to add: 2584
fib to add: 10946
fib to add: 46368
fib to add: 196418
fib to add: 832040
fib to add: 3524578
Es wurden 33 Fibanocci Zahlen berechnet.

Sum of even numbers: 4613732