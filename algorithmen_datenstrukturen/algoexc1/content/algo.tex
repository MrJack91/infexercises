\chapter*{Algorithmus und Datenstruktur - Übungsserie}

\section*{Aufgabe 1}


\section*{Aufgabe 4}

Gerade Fibanocci Zahlen bis 4'000'000.
Anschliessend die Summe davon.

\begin{center}
% Style changes
\small\renewcommand{\arraystretch}{1.4}
% tabular
\small
\rowcolors{1}{tablerowcolor}{tablebodycolor} % mark even rows grey
\begin{tabularx}{0.25\textwidth}{rr}
\rowcolor{tableheadcolor}
\hline
\textbf{\#Fib} & \textbf{Wert} \\
\hline
3 & 2 \\
6 & 8 \\
9 & 34 \\
12 & \num{144} \\
15 & \num{610} \\
18 & \num{2584} \\
21 & \num{10946} \\
24 & \num{46368} \\
27 & \num{196418} \\
30 & \num{832040} \\
33 & \num{3524578} \\
\hline
\rowcolor{tablesubheadcolor}
\textbf{Summe} & \textbf{\num{4613732}}\\
\hline
\end{tabularx}
\end{center}


\begin{table}[ht]
\centering
% \small\renewcommand{\arraystretch}{1.4}  
\rowcolors{1}{tablerowcolor}{tablebodycolor}
%
\captionabove[]{Gerade Fibanocci Zahlen bis 4'000'000.}
\label{tab:IsingModel}
%
%\begin{tabularx}{0.5\textwidth}{r l}
\begin{tabularx}{0.3\textwidth}{r r}
\hline
\rowcolor{tableheadcolor}
\textbf{\#Fib} & \textbf{Wert} \\
\hline
2 & 2 \\
5 & 8 \\
8 & 34 \\
11 & \num{144} \\
14 & \num{610} \\
17 & \num{2584} \\
20 & \num{10946} \\
23 & \num{46368} \\
26 & \num{196418} \\
29 & \num{832040} \\
32 & \num{3524578} \\
\hline
\textbf{Summe} & \textbf{\num{4613732}}\\
\hline
\end{tabularx}
\end{table}